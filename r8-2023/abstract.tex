
\begin{abstract}
    Ventricular tachycardia (VT) poses a persistent challenge in individuals with pre-existing heart conditions, necessitating innovative solutions beyond the limitations of traditional ECG-based methods. In response, this paper presents a groundbreaking approach that integrates multi-sensor fusion, incorporating data from ECG, accelerometer, and gyroscope sensors, with cutting-edge machine learning techniques for context-aware VT detection.

    Drawing upon the PhysioNet CPSC2011 database as a collaborative and open-source data source, our research employs novel deep learning algorithms to enhance the accuracy of VT detection. Results indicate a remarkable improvement, with our approach achieving a detection accuracy of %95%, surpassing the 82% obtained through traditional methods. 
    Furthermore, our methodology enables personalized risk prediction by leveraging individual sensor data patterns.
    
    This research carries significant implications for the realm of cardiac care, offering a paradigm shift in VT management. The fusion of multi-sensor data and advanced machine learning not only facilitates early VT detection but also opens avenues for tailored and personalized management strategies. The potential benefits extend to enhancing patient care in complex cardiac cases, thereby reinforcing the importance of our approach in reshaping the future landscape of cardiac health. As we unveil the potential of context-aware VT detection, this research stands poised to revolutionize the field and contribute to the advancement of personalized healthcare strategies for cardiac patients.

    \textbf{Keywords: ventricular tachycardia, multi-sensor fusion, machine learning, context-awareness, personalized healthcare}
\end{abstract}