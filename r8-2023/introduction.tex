\section{Introduction}
Ventricular tachycardia (VT), a cardiac arrhythmia characterized by a rapid, sustained heartbeat originating in the ventricles, poses a formidable threat to individuals with underlying heart conditions. This arrhythmia, often exceeding 100 beats per minute, disrupts the heart's normal rhythm and can lead to severe complications, including heart failure and sudden cardiac death. 
Despite notable advancements in medical technology, the existing methods for detecting VT, primarily centered around electrocardiogram (ECG) analysis, encounter substantial challenges that compromise their effectiveness~\cite{wellens2001ventricular}\cite{6570512}.      

The foremost limitations of conventional VT detection methods are threefold: limited sensitivity, low specificity, and a lack of contextual understanding. The sensitivity of traditional approaches is marred by their tendency to overlook early-stage VT episodes or erroneously categorize them as benign arrhythmias. This inherent drawback compromises the timely identification of potentially life-threatening situations. Furthermore, the low specificity of these methods results in an alarming rate of false positives, attributing normal physiological variations, noise, or other cardiac conditions to VT events. This issue not only burdens healthcare providers with unnecessary alarms but also contributes to the erosion of trust in the reliability of detection systems. Lastly, the absence of context in current methods overlooks critical individual factors such as posture, activity level, and pre-existing conditions, all of which significantly influence the occurrence and severity of VT episodes.

These limitations underscore the pressing need for a paradigm shift in VT detection—a solution that transcends the shortcomings of current methodologies. The urgency is accentuated by the severe consequences of missed diagnoses and the increasing prevalence of cardiovascular diseases. A more accurate, personalized, and context-aware approach is imperative to enhance the precision of VT detection, paving the way for timely interventions and improved patient outcomes.

Motivated by this imperative, our research endeavors to introduce a novel methodology that harnesses the power of multi-sensor fusion and advanced machine learning algorithms. By integrating data streams from diverse sensors, including ECG, accelerometer, and gyroscope, our approach aims to construct a comprehensive and nuanced representation of the patient's physiological state. The subsequent analysis, facilitated by cutting-edge machine learning techniques, holds the promise of overcoming the limitations of traditional methods.

The pivotal question guiding our investigation is: Can the synergistic integration of multi-sensor fusion and machine learning effectively address the challenges inherent in VT detection, providing accurate, context-aware insights for individuals with pre-existing heart conditions? In answering this question, we envision a transformative impact on cardiac care, where early detection, personalized risk prediction, and real-time insights converge to redefine the standards of patient-centric healthcare, ultimately mitigating the risks associated with VT and improving the quality of life for those vulnerable to cardiac events.

 
